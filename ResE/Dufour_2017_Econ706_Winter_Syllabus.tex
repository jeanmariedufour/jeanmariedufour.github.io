%% \usepackage{times}
%% \usepackage[francais]{babel}


\documentclass[titlepage,11pt,amstex]{article}
%%%%%%%%%%%%%%%%%%%%%%%%%%%%%%%%%%%%%%%%%%%%%%%%%%%%%%%%%%%%%%%%%%%%%%%%%%%%%%%%%%%%%%%%%%%%%%%%%%%%%%%%%%%%%%%%%%%%%%%%%%%%%%%%%%%%%%%%%%%%%%%%%%%%%%%%%%%%%%%%%%%%%%%%%%%%%%%%%%%%%%%%%%%%%%%%%%%%%%%%%%%%%%%%%%%%%%%%%%%%%%%%%%%%%%%%%%%%%%%%%%%%%%%%%%%%
\usepackage{amsmath}
\usepackage{comment}
\usepackage{type1cm}
\usepackage{times}

\setcounter{MaxMatrixCols}{10}
%TCIDATA{OutputFilter=LATEX.DLL}
%TCIDATA{Version=5.50.0.2960}
%TCIDATA{<META NAME="SaveForMode" CONTENT="1">}
%TCIDATA{BibliographyScheme=BibTeX}
%TCIDATA{Created=Wed Aug 13 11:04:47 1997}
%TCIDATA{LastRevised=Monday, January 09, 2017 11:19:41}
%TCIDATA{<META NAME="GraphicsSave" CONTENT="32">}
%TCIDATA{Language=American English}
%TCIDATA{CSTFile=article.cst}

\input{DufourP2}
\input{DufourC1}
\includecomment{Tentative}
\begin{comment}
\newtheorem{theorem}{Theorem}[section]
\newtheorem{acknowledgement}[theorem]{Acknowledgement}
\newtheorem{algorithm}[theorem]{Algorithm}
\newtheorem{assumption}[theorem]{Assumption}
\newtheorem{axiom}[theorem]{Axiom}
\newtheorem{case}[theorem]{Case}
\newtheorem{claim}[theorem]{Claim}
\newtheorem{conclusion}[theorem]{Conclusion}
\newtheorem{condition}[theorem]{Condition}
\newtheorem{conjecture}[theorem]{Conjecture}
\newtheorem{corollary}[theorem]{Corollary}
\newtheorem{criterion}[theorem]{Criterion}
\newtheorem{definition}[theorem]{Definition}
{\theorembodyfont{\normalfont}
\newtheorem{example}[theorem]{Example}
}
\newtheorem{exercise}[theorem]{Exercise}
\newtheorem{lemma}[theorem]{Lemma}
\newtheorem{notation}[theorem]{Notation}
\newtheorem{problem}[theorem]{Problem}
{\theorembodyfont{\normalfont}
\newtheorem{proofth}[theorem]{Proof}
}
\newtheorem{proposition}[theorem]{Proposition}
\newtheorem{remark}[theorem]{Remark}
\newtheorem{solution}[theorem]{Solution}
{\theorembodyfont{\normalfont}
\newtheorem{statement}[theorem]{}
}
\newtheorem{summary}[theorem]{Summary}
\newenvironment{proof}[1][\proofname]{\par
  \normalfont
  \trivlist
  \item[\hskip\labelsep\scshape
    #1{.}]\ignorespaces
}{  \qed\endtrivlist\vspace{\baselineskip} }
\newenvironment{proofnoname}{\par\noindent
  \normalfont}  {\qed\vspace{\baselineskip}}
\newenvironment{sec2subsec}{\renewcommand{\section}{\subsection}}{}
\newenvironment{reflist}{\quad \newline \noindent {\Large \bf References}
  \newline \quad \newline
  \begin{list}{}{\itemindent -.15in \leftmargin .15in \parsep 0in
                 \itemsep 0in} \item \vspace{-.35in} }{\end{list}}
\end{comment}
\input{tcilatex}
\begin{document}

\title{McGill University\\
Department of Economics\\
Econ 706B\\
Special topics in econometrics / Sujets sp\'{e}ciaux d'\'{e}conom\'{e}trie \\
Winter / Hiver 2017\\
Course outline / Syllabus \\
Preliminary / Pr\'{e}liminaire}
\author{Professor / Professeur :\ Jean-Marie Dufour}
\date{January - April 2017 / Janvier - avril 2017\\
Version: \today }
\maketitle

\pagenumbering{arabic}\setcounter{page}{1}

This course has two main objectives

\begin{enumerate}
\item The first one is to review basic statistical and econometric theory at
an advanced level, especially (but not exclusively) in view of developing
tests and confidence sets in econometrics.

\item The second objective consists in studying a number of special topics
of interest for advanced work and research. Methods allowing one to obtain
finite-sample inference procedures as well more reliable large-sample
methods will be emphasized. Among the topics considered, the following ones
will get special attenbtion:

\begin{enumerate}
\item statistical inference methods based on simulation (Monte Carlo tests,
bootstrapping);

\item finite-sample nonparametric methods;

\item projection techniques for the construction of tests and confidence
sets;

\item identification, testability, exogeneity and weak instrument problems
in structural modelling (Highlight theme for 2016-2017);

\item causality analysis in econometrics (Highlight theme for 2016-2017).
\end{enumerate}
\end{enumerate}

\noindent A more detailed lists of topics is available at the end of this
syllabus.

\noindent A list of papers related to the Highlight themes will be supplied
later.

\noindent The course should be especially useful to students who wish to
prepare the a comprehensive Ph.D. examination in Econometrics and eventually
write a dissertation where Econometrics plays an important role.

\begin{center}
\_\_\_\_\_\_
\end{center}

Ce cours comporte deux objectifs principaux. Le premier consiste \`{a} \'{e}%
tudier certains \'{e}l\'{e}ments de th\'{e}orie statistique qui sont
importants pour le d\'{e}veloppement de tests et r\'{e}gions de confiance en 
\'{e}conom\'{e}trie. Le second sera d'\'{e}tudier et appliquer dans diff\'{e}%
rents contextes diverses m\'{e}thodes qui permettent d'obtenir des proc\'{e}%
dures d'inf\'{e}rence valides dans les \'{e}chantillons finis ou encore des
proc\'{e}dures asymptotiques plus fiables. Parmi les sujets qui recevront
une attention particuli\`{e}re, on notera: les techniques de test fond\'{e}%
es sur le recours \`{a} des simulations (tests de Monte Carlo, bootstrap),
l'emploi de techniques de projection pour la construction de tests et r\'{e}%
gions de confiance, les probl\`{e}mes d'instruments faibles en \'{e}conom%
\'{e}trie structurelle, ainsi que diverses techniques non-param\'{e}triques
exactes.

\quad

\begin{center}
\_\_\_\_\_\_
\end{center}

\noindent Documents and other material relevant to the course will be
available from my homepage:

\quad

\begin{center}
\texttt{http://www.jeanmariedufour.com}

\texttt{http://www.jeanmariedufour.org}

\_
\end{center}

\noindent \textbf{Lecture hours}: Monday, 18:05 - 20:55 [plus Thursday,
January 7, 18:05 - 20:55].

Beginning: Monday, January 9, 2017. End: Monday, April 10, 2017.

Exams end on Friday April 28, 2017.

\noindent \textbf{Room}: Leacock 520.

\noindent \textbf{Office hours}: by appointment

\quad

\noindent \textbf{e-mail}: \texttt{jean-marie.dufour@mcgill.ca}

\quad

\noindent The evaluation will be based on three elements (percentage refer
to the entire year's grade):

\begin{enumerate}
\item a mid-term exam: 20\% of the grade;

\item problem sets: 20\% of the grade;

\item short paper and presentation where a recent publised paper on one of
the Highlight topic: 20\% of the grade;

\item a final exam: 40\% of the grade.
\end{enumerate}

\begin{center}
\_\_\_\_\_\_
\end{center}

\noindent L'\'{e}valuation sera bas\'{e}e sur deux \'{e}l\'{e}ments:

\begin{enumerate}
\item un examen intra-semestriel: 20\% de la note;

\item exercices: 20\% de la note;

\item court texte et pr\'{e}sentation sur un article reli\'{e}s \`{a} un des
th\`{e}mes sp\'{e}ciaux choisi pour 2016-2017: 20\% de la note;

\item examen final: 40\% de la note.
\end{enumerate}

\quad \newpage

\noindent In accord with McGill University's Charter of Students' Rights,
students in this course have the right to submit in English or in French any
written work that is to be graded. \newline
\quad \newline
Official statements:

\quad

\noindent McGill University values academic integrity. Therefore all
students must understand the meaning and consequences of cheating,
plagiarism and other academic offences under the Code of Student Conduct and
Disciplinary Procedures (see www.mcgill.ca/students/srr/honest/ for more
information).

\noindent According to Senate regulations, instructors are not permitted to
make special arrangements for final exams. Please consult the calendar,
section 4.7.2.1, General University Information and Regulations, at
www.mcgill.ca .

\quad

\noindent L'universit\'{e} McGill attache une haute importance \`{a} l'honn%
\^{e}tet\'{e} acad\'{e}mique. Il incombe par cons\'{e}quent \`{a} tous les 
\'{e}tudiants de comprendre ce que l'on entend par tricherie, plagiat et
autres infractions acad\'{e}miques, ainsi que les cons\'{e}quences que
peuvent avoir de telles actions, selon le Code de conduite de l'\'{e}tudiant
et des proc\'{e}dures disciplinaires (pour de plus amples renseignements,
veuillez consulter le site www.mcgill.ca/students/srr/honest/ ).

\quad \newpage

\begin{center}
\textbf{Recommended texts / Manuels recommand\'{e}s / }

\quad
\end{center}

(GM) Gouri\'{e}roux, C. et A. Monfort (1989), Statistique et mod\`{e}les 
\'{e}conom\'{e}triques, Volumes 1 et 2. Economica, Paris.

(GM) Gouri\'{e}roux, C. et A. Monfort (1995), Statistics and Econometric
Models, Volumes 1 and 2. Cambridge University Press, Cambridge, U.K..
English translation of previous book.

(U) Ullah, A. (2004), Finite Sample Econometrics, Advanced texts in
Econometrics, Cambridge University Press, Cambridge, U.K.

\quad

\begin{center}
\textbf{Other books used / Autres livres utilis\'{e}}s

\quad
\end{center}

(Am)\ Amemiya, T. (1985), Advanced Econometrics, Harvard University Press,
Cambridge, Massachusetts.

(An)\ Anderson, T. W. (1984), An Introduction to Multivariate Statistical
Analysis, Second Edition, Wiley, New York.

(DM) Davidson, R. et J. G. MacKinnon (1993), Estimation and Inference in
Econometrics, Oxford University Press, Oxford.

(GA)\ Gallant, A. R. (1987), Nonlinear Statistical Models, Wiley, New York.

(GW)\ Gallant, A. R. et H. White (1988), A Unified Theory of Estimation and
Inference for Nonlinear Statistical Models, Basil Blackwell, New York.

(LE)\ Lehmann, E. L. (1983), Theory of Point Estimation, Wiley, New York.

(LT)\ Lehmann, E. L. (1986), Testing Statistical Hypotheses, Second Edition,
Wiley, New York.

(R)\ Rao, C. R. (1973), Linear Statistical Inference and its Applications,
Second Edition, Wiley, New York.

(W)\ White, H. (1984), Asymptotic Theory for Econometricians, Academic
Press, Orlando, Florida.

\newpage

\begin{center}
Homework schedule

\quad

%TCIMACRO{\TeXButton{\footnotesize}{\footnotesize}}%
%BeginExpansion
\footnotesize%
%EndExpansion

$%
\begin{tabular}[t]{|l|l|l|l|l|}
\hline
Week\rule{0cm}{3ex} & \text{Day} & \text{Time }(18:05-20:55) & \text{%
Exercise set to hand in} &  \\ \hline
2 \rule{0cm}{3ex} & \text{Monday} & \text{9 January }2017 &  &  \\ \hline
3 \rule{0cm}{3ex} & \text{Monday} & \text{16 January }2017 & \text{Exercises
1: Models} &  \\ \hline
4 \rule{0cm}{3ex} & \text{Monday} & \text{23 January }2017 & \text{Exercises
2: Decision Theory. }Exercises 3: Information &  \\ \hline
5 \rule{0cm}{3ex} & \text{Monday} & 30 January 2017 & \text{Exercises 4:
Estimation theory. }Exercises 5: Unbiased estimation &  \\ \hline
6 \rule{0cm}{3ex} & \text{Monday} & \text{6 February }2017 & Exercises 6:
General issues in testing theory &  \\ \hline
7 \rule{0cm}{3ex} & \text{Monday} & \text{13 February }2017 & \text{%
Exercises 7: Unbiased and invariant tests} &  \\ \hline
8 \rule{0cm}{3ex} & \text{Monday} & \text{20 February }2017 & Exercises 8:
Confidence sets & \text{Mid-term exam} \\ \hline
9 \rule{0cm}{3ex} & \text{Monday} & \text{27 }February 2017 & Study week & 
\\ \hline
10 \rule{0cm}{3ex} & \text{Monday} & \text{6 March }2017 & \text{Exercises
9-10: }Maximum likelihood method (estimation and tests) &  \\ \hline
11 \rule{0cm}{3ex} & \text{Monday} & \text{13 March }2017 & \text{Exercises
11: }$M$-estimators &  \\ \hline
12 \rule{0cm}{3ex} & \text{Monday} & \text{20 March }2017 & \text{Exercises
12: Methods of moments} &  \\ \hline
13 \rule{0cm}{3ex} & \text{Monday} & \text{27 March }2017 & No courses
(Easter Monday) &  \\ \hline
14 \rule{0cm}{3ex} & \text{Monday} & \text{3 April }2017 & Exercises 13:
Equality constraints. \text{Exercises 14: Prediction and residuals} &  \\ 
\hline
15 \rule{0cm}{3ex} & \text{Monday} & \text{10 April }2017 & \text{Exercises
15: General asymptotic tests} &  \\ \hline
\end{tabular}%
\ $

%TCIMACRO{\TeXButton{\normalsize}{\normalsize}}%
%BeginExpansion
\normalsize%
%EndExpansion
\end{center}

\quad \newpage

\begin{equation*}
\text{{\huge Topics}}
\end{equation*}

The course of this year will focus on the following topics

\begin{enumerate}
\item General statistical and econometric theory

\begin{enumerate}
\item Statistical models

\item Statistical problems
\end{enumerate}

\item Testing and confidence set theory

\item Identification and testability

\item Approaches to inference

\begin{enumerate}
\item Finite-sample inference

\item Asymptotic inference
\end{enumerate}

\item Quantiles, Lorenz curves and inequality

\begin{enumerate}
\item Distribution function functions and quantiles

\item Lorenz curves

\item Inequality measures
\end{enumerate}

\item Special topics in inference

\begin{enumerate}
\item Simulation-based methods

\begin{enumerate}
\item Monte Carlo tests

\item Boostrapping
\end{enumerate}

\item Methods for dealing with nuisance parameters

\begin{enumerate}
\item Bounds methods

\item Two-stage confidence procedures
\end{enumerate}

\item $C(\alpha )$ tests
\end{enumerate}

\item Causality

\begin{enumerate}
\item General theories of causality

\item Causality in statistical models

\item Causality in static statistical models

\item Direct, indirect and total effects

\item Treatment effects and policy analysis

\item Causality in time series

\begin{enumerate}
\item Wiener-Granger causality

\item Multiple horizon causality
\end{enumerate}

\item Causality in macroeconomics
\end{enumerate}
\end{enumerate}

\newpage

\begin{equation*}
\text{{\huge Detailed topics}}
\end{equation*}

\begin{center}
{\Huge A. General econometric theory}
\end{center}

\quad

\section{\textbf{Philosophy of science and statistics \label{Philosophy of
science and statistics}}}

\begin{enumerate}
\item \label{Objectives}Objectives of scientific knowledge

\item \label{Criteria}Criteria for evaluating theories and models
\end{enumerate}

\section{Inference \textbf{techniques \label{Inference techniques}}}

\begin{enumerate}
\item Statistical models

\begin{enumerate}
\item \label{Notion of statistical model}Notion of statistical model

\item \label{Important econometric models}Important econometric models
\end{enumerate}

\item \label{Statistical models}Statistical problems

\begin{enumerate}
\item \label{Statistics and decision theory}Statistical problems as decision
problems

\item \label{R: Important statistical problems}Review of important
statistical problems

\begin{enumerate}
\item Estimation

\item Tests

\item Confidence regions

\item Multiple tests and simultaneous inference

\item Prediction

\item Model selection
\end{enumerate}
\end{enumerate}

\item Information and identification

\begin{enumerate}
\item Sufficient and ancillary statistics

\item Information

\item Identification
\end{enumerate}

\item Estimation

\begin{enumerate}
\item Criteria for evaluating estimators

\item Unbiased estimation

\item Some general estimation methods

\begin{enumerate}
\item Maximum likelihood

\item M-estimators

\item Instrumental variables

\item Methods of moments

\item Minimum distance
\end{enumerate}
\end{enumerate}

\item Testing

\begin{enumerate}
\item Basic concepts: level, size, power, conservative test, liberal test

\item Optimal tests and Neyman-Pearson theorem

\item Important classes of tests

\begin{enumerate}
\item Similar tests

\item Unbiased tests

\item Invariant tests
\end{enumerate}

\item Some general methods for building tests

\begin{enumerate}
\item Likelihood ratio

\item Wald tests

\item Score-based procedures [Rao, Lagrange multiplier, Neyman's $C(\alpha )$%
]

\item Union-intersection methods
\end{enumerate}
\end{enumerate}

\item Confidence regions

\begin{enumerate}
\item Basic concepts

\item Pivotal functions

\item Duality between tests and confidence regions
\end{enumerate}

\item Multiple tests and simultaneous inference

\item Prediction and residuals

\item Model selection

\item Bayesian approach
\end{enumerate}

\section{Distributional problems and finite-sample analysis \label%
{Distributional problems and finite-sample analysis}}

\begin{enumerate}
\item Asymptotic theory and its limitations

\begin{enumerate}
\item Review of basic asymptotic notions and results

\item Asymptotic expansions

\item Limitations of asymptotic theory
\end{enumerate}

\item Invariance problems in nonlinear models

\item Techniques for building finite-sample inference procedures

\begin{enumerate}
\item Analytical derivation of distributions

\begin{enumerate}
\item Exact distributions of quadratic forms in Gaussian variables

\item Imhof's algorithm
\end{enumerate}

\item Elimination of nuisance parameters

\begin{enumerate}
\item Conditioning

\item Transformations
\end{enumerate}

\item Bounds

\item Projection

\item Randomization
\end{enumerate}

\item Theory of Monte Carlo tests

\item Bootstrap
\end{enumerate}

\section{Static and dynamic regressions\textbf{\ \label{Static and dynamic
regressions}}}

\begin{enumerate}
\item Nonregular problems in classical linear regressions

\begin{enumerate}
\item Linear models with non-normal disturbances

\item Confidence intervals for ratios of coefficients, Fieller's method

\item Linear models with exact collinearity

\item Tests of nonlinear hypotheses

\item Nonlinear restrictions

\item Tests of multiple hypotheses
\end{enumerate}

\item Specification tests and analysis of residuals

\begin{enumerate}
\item Normality of errors

\item Heteroskedasticity

\item Autocorrelation

\item Outliers
\end{enumerate}

\item Linear regressions with autocorrelated errors

\item Multiple equation regression models

\begin{enumerate}
\item Multivariate linear regressions (MLR)

\item Seemingly unrelated regressions (SURE)
\end{enumerate}

\item Regressions with heteroskedastic errors

\item Inference problems in dynamic models

\begin{enumerate}
\item Review of technical difficulties

\item Exact inference in dynamic models
\end{enumerate}

\item Structural change analysis

\item Nonlinear regressions
\end{enumerate}

\section{Identification and structural models \label{Identification and
structural models}}

\begin{enumerate}
\item Simultaneous equations and identification

\item Inference problems associated with identification.

\begin{enumerate}
\item Impossibility theorems

\item Weak instruments
\end{enumerate}

\item Exact inference in structural models

\item Methods adapted to weak instruments

\item Nonlinear structural models

\item Generalized method of moments
\end{enumerate}

\section{Causality and multivariate time series models \label{Causality and
multivariate time series models}}

\begin{enumerate}
\item \label{Multivariate time series models}Multivariate time series models

\item \label{Causality in econometrics}General notions on causality in
econometrics

\item \label{Causality in multivariate time series models}Causality in
multivariate time series models
\end{enumerate}

\section{Nonparametric methods \label{Nonparametric methods}}

\begin{enumerate}
\item Signs, ranks and permutations

\item Location tests

\item Tests against serial dependence

\item Conditional independence tests

\item Goodness-of-fit tests
\end{enumerate}

\begin{center}
\newpage \textbf{Sujets}

\quad
\end{center}

\section{\textbf{Philosophie des sciences et statistique}}

\begin{enumerate}
\item Objectifs de la connaissance scientifique

\item Crit\`{e}res servant \`{a} \'{e}valuer les th\'{e}ories et les mod\`{e}%
les
\end{enumerate}

\section{\textbf{Techniques d'inf\'{e}rence}}

\begin{enumerate}
\item Mod\`{e}les statistiques

\begin{enumerate}
\item La notion de mod\`{e}les statistique

\item Quelques mod\`{e}les \'{e}conom\'{e}triques importants
\end{enumerate}

\item \label{Statistical problems}Probl\`{e}mes statistiques

\begin{enumerate}
\item Les probl\`{e}mes statistiques comme probl\`{e}mes de d\'{e}cision

\item Revue des principaux probl\`{e}mes statistiques

\begin{enumerate}
\item Estimation

\item Tests

\item R\'{e}gions de confiance

\item Tests multiples et inf\'{e}rence simultan\'{e}e

\item Pr\'{e}vision

\item Choix de mod\`{e}les
\end{enumerate}
\end{enumerate}

\item \label{Information and sufficiency}Information et identification

\begin{enumerate}
\item \label{Sufficiency}Notions de statistique exhaustive et de statistique
libre

\item \label{Information}Information

\item \label{Identification}Identification
\end{enumerate}

\item \label{Estimation}Estimation

\begin{enumerate}
\item Crit\`{e}res pour les estimateurs

\item Estimation sans biais

\item Quelques m\'{e}thodes g\'{e}n\'{e}rales d'estimation

\begin{enumerate}
\item Maximum de vraisemblance

\item M-estimateurs

\item Variables instrumentales

\item M\'{e}thodes de moments

\item Distance minimale
\end{enumerate}
\end{enumerate}

\item \label{Testing}Tests

\begin{enumerate}
\item Concepts de base: niveau, taille, puissance, tests conservateurs,
tests lib\'{e}raux

\item Tests optimaux et th\'{e}or\`{e}me de Neyman-Pearson

\item Classes importantes de tests

\begin{enumerate}
\item Tests $\alpha $-semblables

\item Tests sans biais

\item Tests invariants
\end{enumerate}

\item Quelques m\'{e}thodes g\'{e}n\'{e}rales pour construire des tests

\begin{enumerate}
\item Quotient de vraisemblance

\item Tests de Wald

\item Tests fond\'{e}s sur la vraisemblance [Rao, multiplicateur de
Lagrange, $C(\alpha )$ de Neyman]

\item M\'{e}thodes d'union-intersection
\end{enumerate}
\end{enumerate}

\item \label{Confidence regions}R\'{e}gions de confiance

\begin{enumerate}
\item Concepts de base

\item Notion de fonction pivotale

\item Dualit\'{e} entre tests et r\'{e}gions de confiance
\end{enumerate}

\item \label{Simultaneous inference}Tests multiples et inf\'{e}rence simultan%
\'{e}e

\item \label{Prediction}Pr\'{e}vision et r\'{e}sidus

\item \label{Model selection}Choix de mod\`{e}les

\item \label{Bayesian approach}Approche bay\'{e}sienne
\end{enumerate}

\section{Probl\`{e}mes distributionnels et analyse \`{a} distance finie}

\begin{enumerate}
\item \label{Asymptotic theory and its limitations}La th\'{e}orie
asymptotique et ses limitations

\begin{enumerate}
\item \label{Asymptotic theory}Rappels de th\'{e}orie asymptotique

\item \label{Asymptotic expansions}Expansions asymptotiques

\item \label{Limitations of asymptotic theory}Limitations de la th\'{e}orie
asymptotique
\end{enumerate}

\item \label{Test invariance}Probl\`{e}mes d'invariance de tests dans les mod%
\`{e}les non lin\'{e}aires

\item \label{Finite-sample inference techniques}Techniques pour la mise au
point de proc\'{e}dures d'inf\'{e}rence \`{a} distance finie

\begin{enumerate}
\item \label{Exact analytical distributions}D\'{e}rivation de distributions
analytiques exactes

\begin{enumerate}
\item \label{Gaussian quadratic forms}Distributions exactes de formes
quadratiques dans le cas gaussien

\item \label{Imhof algorithm}Algorithme d'Imhof
\end{enumerate}

\item \label{Nuisance parameter elimination}\'{E}limination des param\`{e}%
tres du nuisance

\begin{enumerate}
\item \label{Conditioning}Conditionnement

\item \label{Transformations}Transformations
\end{enumerate}

\item \label{Bounds}Bornes

\item \label{Projection}Projection

\item \label{Randomization}Randomisation
\end{enumerate}

\item \label{Monte Carlo tests}Th\'{e}orie des tests de Monte Carlo

\item \label{Bootstrapping}Bootstrap
\end{enumerate}

\section{\textbf{Mod\`{e}les de r\'{e}gression statiques et dynamiques}}

\begin{enumerate}
\item \label{Nonregular problems in linear regression}Probl\`{e}mes non r%
\'{e}guliers dans le mod\`{e}le lin\'{e}aire classique

\begin{enumerate}
\item \label{Linear models with non-normal errors}Mod\`{e}les lin\'{e}aires
avec erreurs non-normales

\item \label{Ratios of coefficients}Intervalles de confiance pour des ratios
de coefficients, m\'{e}thode de Fieller

\item \label{Linear models with collinearity}Mod\`{e}les lin\'{e}aires avec
collin\'{e}arit\'{e} exacte

\item \label{Nonlinear hypotheses}Tests d'hypoth\`{e}ses non lin\'{e}aires

\item \label{Nonlinear constraints}Contraintes non-lin\'{e}aires

\item \label{Multiple hypotheses}Tests d'hypoth\`{e}ses multiples
\end{enumerate}

\item \label{Specification tests}Tests de sp\'{e}cification et analyse de r%
\'{e}sidus

\begin{enumerate}
\item \label{Normality tests}Normalit\'{e} des erreurs

\item \label{Heteroskedasticity tests}H\'{e}t\'{e}rosc\'{e}dasticit\'{e}

\item \label{Autocorrelation tests}Autocorr\'{e}lation

\item \label{Outlier tests}Observations \`{a} l'\'{e}cart
\end{enumerate}

\item \label{Linear regressions with autocorrelated errors}Mod\`{e}les de r%
\'{e}gression avec erreurs autocorr\'{e}l\'{e}es

\item \label{Multiple equations linear regressions}Mod\`{e}les de r\'{e}%
gression \`{a} plusieurs \'{e}quations

\begin{enumerate}
\item \label{MLR}Mod\`{e}les de r\'{e}gression multivari\'{e}s (MLR)

\item \label{SURE}R\'{e}gressions empil\'{e}es (SURE)

\item \label{CAPM}Applications \`{a} des mod\`{e}les d'\'{e}valuation de
prix d'actifs financiers (CAPM)
\end{enumerate}

\item \label{Regressions with heteroskedasticity}R\'{e}gressions avec
erreurs h\'{e}t\'{e}rosc\'{e}dastiques

\item \label{Inference in dynamic models}Probl\`{e}mes d'inf\'{e}rence dans
les mod\`{e}les dynamiques

\begin{enumerate}
\item \label{Technical difficulties}Revue des difficult\'{e}s techniques

\item \label{Exact inference methods in dynamic models}Inf\'{e}rence exacte
dans les mod\`{e}les dynamiques
\end{enumerate}

\item \label{Structural change analysis}Probl\`{e}mes d'analyse du
changement structurel

\item \label{Nonlinear regression}Mod\`{e}les de r\'{e}gression non-lin\'{e}%
aires
\end{enumerate}

\section{Mod\`{e}les structurels}

\begin{enumerate}
\item \label{Simultaneous equations}\'{E}quations simultan\'{e}es lin\'{e}%
aires et identification

\item \label{Identification and inference}Probl\`{e}mes d'inf\'{e}rence reli%
\'{e}s \`{a} l'identification.

\begin{enumerate}
\item \label{Impossibility theorems}Th\'{e}or\`{e}mes d'impossibilit\'{e}

\item \label{Weak instruments}Instruments faibles
\end{enumerate}

\item \label{Exact inference in structural models}Inf\'{e}rence exacte dans
les mod\`{e}les structurels

\item \label{Methods adapted to WI}M\'{e}thodes d'inf\'{e}rence adapt\'{e}es
aux instruments faibles

\item \label{Nonlinear structural models}Mod\`{e}les structurels non lin\'{e}%
aires

\item \label{GMM}M\'{e}thode des moments g\'{e}n\'{e}ralis\'{e}s
\end{enumerate}

\section{Causalit\'{e} et mod\`{e}les de s\'{e}ries chronologiques multivari%
\'{e}s}

\begin{enumerate}
\item Mod\`{e}les de s\'{e}ries chronologiques multivari\'{e}s

\item G\'{e}n\'{e}ralit\'{e}s sur la causalit\'{e} en \'{e}conom\'{e}trie

\item Causalit\'{e} dans les mod\`{e}les de s\'{e}ries chronologiques
multivari\'{e}s
\end{enumerate}

\section{M\'{e}thodes non-param\'{e}triques}

\begin{enumerate}
\item \label{Sign and rank tests}G\'{e}n\'{e}ralit\'{e}s sur les tests de
signes, de rangs et de permutations

\item \label{Distribution-free location tests}Tests de localisation

\item \label{Distribution-free serial dependence tests}Tests contre la d\'{e}%
pendance s\'{e}rielle

\item \label{Distribution-free orthogonality tests}Tests d'orthogonalit\'{e}

\item \label{Goodness-of-fit tests}Tests d'ajustement
\end{enumerate}

\newpage

\begin{center}
\textbf{Readings and main references / Lectures et r\'{e}f\'{e}rences
principales }

\quad
\end{center}

\quad

The symbol * represents required readings. Topics for which no reference is
provided will covered in class-notes. Photocopied lecture notes also
constitute required reading.

Le symbole * repr\'{e}sente des lectures obligatoires. Les sujets pour
lesquels aucune r\'{e}f\'{e}rence n'appara\^{\i}t seront couverts au moyen
de notes de cours. Les notes de cours photocopi\'{e}es constituent des
lectures obligatoires.

\quad

\noindent \textbf{\ref{Philosophy of science and statistics}. Philosophy of
science and statistics / Philosophie des sciences et statistique}

\quad

\begin{description}
\item[\quad ] \textbf{\ref{Objectives} - \ref{Criteria}}.\quad \cite%
{Dufour(2000)}, \cite{Dufour(2001)}.
\end{description}

\quad

\noindent \textbf{\ref{Inference techniques}}. \textbf{Inference techniques
/ Techniques d'inf\'{e}rence}

\quad

\begin{description}
\item[\quad ] \textbf{\ref{Statistical models}}. * GM, Ch. 1.

\item[\quad ] \textbf{\ref{Statistics and decision theory}}.\quad * GM, Ch.
2.

\item[\quad ] \textbf{\ref{Information and sufficiency}}.* GM, Ch. 3. \cite%
{Basu(1977)}.

\item[\quad ] \textbf{\ref{Estimation}}.\quad GM, Ch. 5. 6.

\item[\quad ] \textbf{\ref{Testing}}.\quad * GM, Ch. 14, 15, 16.

\item[\quad ] \textbf{\ref{Confidence regions}}.\quad * GM, Ch. 20.

\item[\quad ] \textbf{\ref{Simultaneous inference}}.\quad * GM, Ch. 19.

\item[\quad ] \textbf{\ref{Prediction}}.\quad GM, Ch. 11.

\item[\quad ] \textbf{\ref{Model selection}}.\quad GM, Ch. 22.

\item[\quad ] \textbf{\ref{Bayesian approach}}.\quad GM, Ch. 4, 12
\end{description}

\quad

\noindent \textbf{\ref{Distributional problems and finite-sample analysis}.
Distributional problems and finite-sample analysis / Probl\`{e}mes
distributionnels et analyse \`{a} distance finie}

\quad

\begin{description}
\item[\quad ] \textbf{\ref{Exact analytical distributions}. *}\cite[Chapters
1 and 2]{Ullah(2004)}.

\item \textbf{\ref{Asymptotic theory and its limitations}. *}\cite%
{Dufour(2000)}, *\cite{Dufour(2001)}, \cite{Dufour(2003)}, *\cite%
{Bahadur-Savage(1956)}.

\item \textbf{\ref{Test invariance}}.\quad *\cite{Dufour-Dagenais(1991)}, 
\cite{Dufour-Dagenais(1992)}, \cite{Dufour-Dagenais(1992b)}, \cite%
{Dufour-Dagenais(1994)}.

\item[\quad ] \textbf{\ref{Bounds}}. *\cite{Dufour(1990)}.

\item[\quad ] \textbf{\ref{Projection}}.\quad *\cite%
{Abdelkhalek-Dufour(1998)}.

\item[\quad ] \textbf{\ref{Monte Carlo tests}}.\quad *\cite%
{Dufour-Khalaf(2001)}, *\cite{Dufour(2006)}.
\end{description}

\quad

\noindent \textbf{\ref{Static and dynamic regressions}. Static and dynamic
regressions / Mod\`{e}les de r\'{e}gression statiques et dynamiques}

\quad

\begin{description}
\item[\quad ] *\cite[Chapter 5 and 6]{Ullah(2004)}.

\item \textbf{\ref{Ratios of coefficients}}.\quad *\cite{Dufour(1997)}.

\item[\quad ] \textbf{\ref{Linear models with collinearity}}.\quad *\cite%
{Dufour(1982)}.

\item[\quad ] \textbf{\ref{Nonlinear hypotheses}}.\quad *\cite{Dufour(1989)}.

\item[\quad ] \textbf{\ref{Multiple hypotheses}}.\quad *\cite{Dufour(1989)}.

\item[\quad ] \textbf{\ref{Normality tests}}.\quad \cite%
{Dufour-Farhat-Gardiol-Khalaf(1998)}.

\item[\quad ] \textbf{\ref{Heteroskedasticity tests}}.\quad \cite%
{Dufour-Khalaf-Bernard-Genest(2004)}.

\item[\quad ] \textbf{\ref{MLR}}.\quad \cite{Dufour-Khalaf(2002b)}.

\item[\quad ] \textbf{\ref{SURE}}.\quad \cite{Dufour-Khalaf(2002)}, \cite%
{Dufour-Khalaf(2001d)}, \cite{Dufour-Torres(1998)}.

\item \textbf{\ref{CAPM}}.\quad *\cite{Dufour-Khalaf-Beaulieu(2003)}, *\cite%
{Beaulieu-Dufour-Khalaf(2006)}, *\cite{Beaulieu-Dufour-Khalaf(2007)}, .

\item[\quad ] \textbf{\ref{Regressions with heteroskedasticity}}.\quad \cite%
{Dufour(1991)}, \cite{Dufour-Mahseredjian(1993)}.

\item[\quad ] \textbf{\ref{Inference in dynamic models}}.\quad \cite%
{Dufour(1990)}, *\cite{Dufour-King(1991)}, *\cite{Dufour-Kiviet(1998)}, \cite%
{Kiviet-Dufour(1997)}, \cite{Dufour-Torres(1998)}, \cite{Dufour-Torres(2000)}%
.
\end{description}

\quad

\noindent \textbf{\ref{Identification and structural models}}. \textbf{%
Identification and structural models / Identification et mod\`{e}les
structurels}

\quad

\begin{description}
\item[\quad ] \textbf{\ref{Simultaneous equations}}. *\cite[Chapter 7]%
{Ullah(2004)}.

\item \textbf{\ref{Identification and inference}}. *\cite{Dufour(2003)}, *%
\cite{Stock-Wright-Yogo(2002)}, *\cite{Dufour(1997)}.

\item \textbf{\ref{Exact inference in structural models}}. *\cite%
{Dufour(1997)}, *\cite{Dufour-Jasiak(2001)}, *\cite{Dufour-Taamouti(2005)}.

\item \textbf{\ref{Methods adapted to WI}}. *\cite{Kleibergen(2002)}, *\cite%
{Moreira(2003)}.
\end{description}

\quad

\noindent \textbf{\ref{Nonparametric methods}}. \textbf{Causality and
multivariate time series models / Causalit\'{e} te mod\`{e}les de s\'{e}ries
chronologiques multivari\'{e}s}

\quad

\begin{description}
\item \ref{Causality in econometrics}. \cite{Causality in econometrics}. 
\cite{Pearl(2000)}

\item \ref{Causality in multivariate time series models}. *\cite%
{Dufour-Renault(1998)}, \cite{Dufour-Pelletier-Renault(2006)}, \cite%
{Boudjellaba-Dufour-Roy(1992)}.
\end{description}

\quad

\noindent \textbf{\ref{Nonparametric methods}}. \textbf{Nonparametric
methods / M\'{e}thodes non-param\'{e}triques}

\textbf{\quad }

\begin{description}
\item[\quad ] \textbf{\ref{Sign and rank tests}}. \cite%
{Dufour-Lepage-Zeidan(1982)}.

\item[\quad ] \textbf{\ref{Distribution-free location tests}}. \cite%
{Dufour-Hallin(1990)}.

\item[\quad ] \textbf{\ref{Distribution-free serial dependence tests}}. \cite%
{Dufour(1981)}, \cite{Dufour-Roy(1985)}, \cite{Dufour-Roy(1986)}, \cite%
{Dufour-Roy(1986b)}, \cite{Dufour-Hallin(1987)}, \cite%
{Dufour-Hallin-Mizera(1998)}, \cite{Dufour-Farhat-Hallin(2006)}.

\item[\quad ] \textbf{\ref{Distribution-free orthogonality tests}}. \cite%
{Campbell-Dufour(1991)}, \cite{Campbell-Dufour(1995)}, \cite%
{Campbell-Dufour(1997)}.

\item[\quad ] \textbf{\ref{Goodness-of-fit tests}}. \cite%
{Dufour-Farhat(2001)}, \cite{Dufour-Farhat(2001b)}.
\end{description}

\quad

\quad \newpage

\begin{center}
{\Huge B. Special topics}
\end{center}

\quad

\begin{enumerate}
\item Finite-sample techniques in econometrics

\item Simulation-based inference in econometrics

\item Identification problems in econometrics

\begin{enumerate}
\item Identification theory

\item Exogeneity tests

\item Statistical inference
\end{enumerate}

\item Causality in econometrics

\item Multivariate time series models: statistical analysis, forecasting,
policy analysis

\item Statistical methods for structural dynamic general equilibrium models

\item Econometric analysis of the capital asset pricing models (CAPM)

\item Distribution-free and robust methods in time series and econometrics

\item Statistical analysis for heavy-tailed distribution

\item Volatility modelling

\item Statistical analysis of poverty and inequality

\item Structural change analysis
\end{enumerate}

\begin{comment}

\newpage \textbf{Bibliographie g\'{e}n\'{e}rale / General bibliography}

\begin{center}
\quad
\end{center}

\quad

\bbibunitu\bbibunit

\ebibunit\ebibunitu

\quad

\end{comment}

\newpage

\bibliographystyle{ECONOMETRICA}
\bibliography{refer1}

\end{document}
